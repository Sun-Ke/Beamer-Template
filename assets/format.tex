\documentclass{ctexbeamer}
%\tiny\scriptsize\footnotesize\small\normalsize\large\Large\LARGE \huge\Huge
\useoutertheme{metropolis}
\useinnertheme{metropolis}
%\usefonttheme{metropolis}
\usecolortheme{crane}
\setbeamertemplate{frame numbering}[fraction]		% 当前页/总页数
%\setbeamertemplate{headline}{}						% 不显示最上的导航栏
%\setbeamertemplate{navigation symbols}{}			% 不显示右下的导航图标
\setbeamercovered{transparent=15}					% pause 透明度

%\usefonttheme{professionalfonts} % using non standard fonts for beamer
\usefonttheme{serif} % default family is serif

\setbeamerfont{footnote mark}{size=\scriptsize}		% 脚注符号大小
\setbeamerfont{footnote}{size=\footnotesize}		% 脚注大小
\setbeamerfont{title}{shape=\itshape,family=\rmfamily}
\setbeamerfont{frametitle}{size=\LARGE}

\newcommand{\Sub}{$>>>$}

\usepackage{tikz}
\usetikzlibrary{arrows,patterns,plotmarks,shapes,snakes,er,3d,automata,backgrounds,topaths,trees,petri,mindmap}

\XeTeXlinebreaklocale "zh"							% 表示用中文的断行
\XeTeXlinebreakskip = 0pt plus 1pt					% 多一点调整的空间


\usefonttheme[onlylarge]{structuresmallcapsserif}
\usefonttheme[onlysmall]{structurebold}
%\setbeamercolor{title}{fg=red!80!black,bg=red!20!white}

\def\hilite<#1>{%
\temporal<#1>{\color{gray}}{\color{blue}}%
{\color{blue!45}}}
\renewcommand{\raggedright}{\leftskip=0pt \rightskip=0pt plus 0cm}
\raggedright

\usepackage{pstricks,pst-tree,pst-node,pst-text}
\usepackage{multicol}
\usepackage{amsmath}
\usepackage{listings}
\usepackage{tabularx}

\definecolor{commentcolor}{RGB}{85,139,78}
\definecolor{stringcolor}{RGB}{206,145,108}
\definecolor{keywordcolor}{RGB}{34,34,250}
\definecolor{backcolor}{RGB}{220,220,220}


% 代码环境
\lstset{
%backgroundcolor = \color{gray!10},
numbers=left, 									% 左侧行号,删除本行则无行号
numbersep=8pt,                   				% 行号与代码的间距
%flexiblecolumns=true,             				% 加上后tabsize不准确
numberstyle=\scriptsize, 	
stepnumber=1,                   				% 行号出现的间隔数,是1的话每行都标注行号
basicstyle=\linespread{1}\small\ttfamily, 		% 代码基本设置,行间距,字体大小
commentstyle=\color{commentcolor},  			% 注释
keywordstyle=\bfseries\color{keywordcolor},  	% 关键词加粗
stringstyle=\color{red},    					% 字符串
%escapeinside=&&,								% 逃逸字符,基本不用
extendedchars=true,           					% lets you use non-ASCII characters; for 8-bits encodings only, does not work with UTF-8
frame=shadowbox, % trBL, 
rulesepcolor= \color{red!20!green!20!blue!20},	% 阴影浅灰色
%xleftmargin=2em,								
%xrightmargin=2em,
%aboveskip=1em,
tabsize=4,										% tab的空格数
breaklines=true,  							 	% 自动换行,建议不要写太长的行
%title=\lstname	% show the filename of files included with \lstinputlisting; also try caption instead of title
}

%\setmonofont{Courier New}

\setbeamertemplate{blocks}[rounded][shadow=true]
\setbeamercolor{block body alerted}{bg=alerted text.fg!10}
\setbeamercolor{block title alerted}{bg=alerted text.fg!20, fg=red}
\setbeamercolor{block body}{bg=structure!10}
\setbeamercolor{block title}{bg=structure!20, fg=red}
\setbeamercolor{block body example}{bg=green!10}
\setbeamercolor{block title example}{bg=green!20}

\usepackage{colortbl}

%修改比例16:9
\usepackage[orientation=landscape,size=custom,width=16,height=9,scale=0.36,debug]{beamerposter}

%伪代码设置
\usepackage[lined]{algorithm} 
\usepackage{amsthm}
\usepackage[noend]{algpseudocode}

\algnewcommand{\algorithmicand}{\textbf{ and }}
\algnewcommand{\algorithmicor}{\textbf{ or }}
\algnewcommand{\Or}{\algorithmicor}
\algnewcommand{\AND}{\algorithmicand}
\algnewcommand{\Break}{\textbf{break} }
\makeatletter
\def\BState{\State\hskip-\ALG@thistlm}
\makeatother
